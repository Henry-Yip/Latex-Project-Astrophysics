\documentclass{article}
\usepackage{amsmath}
\usepackage{amssymb}
\usepackage [utf8] {inputenc}
\title{Introductory Astrophysics (PHYS08050) Notes}
\author{Henry Yip}
\begin{document}
\maketitle
\begin{abstract}
This set of notes is made with reference to the Astrophysics Coursebook by The University of Edinburgh. I have gained permission from Professor Catherine Heymans to put my modified notes in my website.
\end{abstract}
\section{Definition}
\begin{itemize}
\item \textbf{Luminosity}
\begin{itemize}
\item The total amount of energy an object radiates each second
\end{itemize}
\item \textbf{Flux}
\begin{itemize}
\item The energy per second flowing through each square metre
\item Defined as the \textbf{Apparent Brightness}
\end{itemize}
\item \textbf{Surface Brightness }
\begin{itemize}
\item The flux of light coming each unit solid angle on the sky (Refer to the back for the definition of \textbf{steradians})
\end{itemize}
\end{itemize}
\section{ $\frac{3}{2}$ law}
\begin{itemize}
\item We can define the formula below:
\begin{equation}
D_\text{max}=\left(\frac{L}{4 \pi F_\text{min}} \right) ^{\frac{1}{2}}
\end{equation}
\item We can hence define $V_\text{max}$ as:
\begin{align}
V_\text{max}&=\frac{4}{3} \pi {D_\text{Max}}^3 \\
&=\frac{1}{6 {\pi}^{\frac{1}{2}}} \left(\frac{L}{F_\text{min}}\right)^{\frac{3}{2}}
\end{align}
With a number density of $n$ stars per unit volume\footnote{Here, we have assumed a constant density, which is rather fishy in my opinion}, we have:
\begin{equation}
N=\frac{n}{6 {\pi}^{\frac{1}{2}}} \left(\frac{L}{F_\text{min}}\right)^{\frac{3}{2}}
\end{equation}
Where $N$ is the total number of stars.
\item In reality things don't work like this. The slope becomes $\frac{1}{2}$ for low fluxes. Why?
\begin{itemize}
\item The likely reason is that the space is not infinite. There are fewer faint stars than our formula predicted \footnote{Why however, what happens to stars with high flux?}
\end{itemize}
\item The reasons why the flattening off the star counts is gradual and not sharp:
\begin{itemize}
\item  The Milky Way star system probably does not have a hard edge, but fades out gradually
\item  Stars are not all the same luminosity, so stars near the edge actually appear at a range
of fluxes
\end{itemize}
\end{itemize}
\section{Important Symbols}
Below we introduce some of the most important symbols in Astrophysics.
\begin{table}[h]
  \begin{center}
    \begin{tabular}{c|c|c}
      \textbf{Symbol} & \textbf{Name}&\textbf{Definition}\\
      \hline
      $\theta^o$ & Degree &\\
      $\theta'$ & Arcminute & $\frac{1}{60}$ of a degree\\
      $\theta"$ & Arcsecond & $\frac{1}{60}$ of an arcminute
      \end{tabular}
  \end{center}
\end{table}
\begin{itemize}
\item We have 
\begin{align}
1\text{rad}&= {\frac{180}{\pi}}^{\circ}\\
&= {\frac{(180)(60)}{\pi}} \text{arcminutes}\\
&={\frac{(180)(3600)}{\pi}} \text{arcseconds}
\end{align}
Rearranging, we also obtain:
\begin{align}
1 \text{arcsecond}&=\frac{1}{\frac{(180)(3600)}{\pi}} \text{rad}\\
&=4.848...*10^{-6}\text{rad}\\
&=4.85  \text{micro-radians}
\end{align}
Meanwhile, 1 steradian (sr) is equal to:
\begin{equation}
\left(\frac{180}{\pi} \right)^2 \text{sq degrees}
\end{equation}
\end{itemize}
\section{Angular resolution}
\begin{itemize}
\item By some complex analysis, we obtain the formula:
\begin{equation}
\theta=\frac{1.22 \lambda}{D}
\end{equation}
\item Factors that affect angular resolution
\begin{itemize}
\item Optical distortions
\item Atmospheric turbulence
\end{itemize}
\end{itemize}
\section{Brightness of the sky}
Assuming stars have constant number density $n$. Consider a shell of thickness $dR$ at distance $R$ from us. The number of stars in this shell is hence $N= 4\pi n  R^2 dR$. Since each star produces a flux of  
\begin{equation}
F=\frac{L}{4\pi R^2}
\end{equation}
The total flux is therefore:
\begin{align}
F_\text{tot}&=NF\\
&=\frac{NL}{4\pi R^2}
\end{align}
With the definition of surface brightness, 
\begin{equation}
B=\frac{nLdR}{4 \pi}
\end{equation}
If we sum up the total brightness by integration, we can see that it doesn't converge, so there's something wrong with our approximations.
\section{Dust}
\begin{itemize}
\item The space between stars is not empty, they are filled with dust.
\item Let $x$ be the distance travelled by the dust. Consider that when $dF=0$, $dx=0$. With other estimations, we have:
\begin{equation}
\frac{dF}{F}=-kdx
\end{equation}
Hence,
\begin{equation}
\frac{F}{F_o}=e^{-kx}
\end{equation}
\item We define $x_e=\frac{1}{k}$, which is the distance over which the flux changes by a factor or $\frac{1}{e}$.
Hence, 
\begin{equation}
\frac{F}{F_o}=e^{-\frac{x}{x_e}}
\end{equation}

\item For example, if every 2 kly light is reduced by a factor of $e$, at 6 kly, light is reduced by a factor of $e^3$.\footnote{THIS PART REVISE MORE}
\item Longer wavelength light has a longer scale length as it is absorbed less.
\item The size of cosmic dust grains is similar to the wavelength of visible light, which is why extinction is so sensitive to wavelength - much larger waves sail past and hardly notice
the dust; much smaller waves are simply blocked.
\end{itemize}
\section{Cosmic Distances}
\begin{itemize}

\item The radius of the earth is denoted by $R_E$, which is defined as 1 A.U. (Astronomical Unit)

\item Angle between direction of Sun and direction Venus is called the angle of elongation ($\theta_E$)

\item When $\theta_E$ is maximum, the line joining Venus and Earth is perpendicular to the line joining Venus and Sun \footnote{Why?}. We can hence measure the distance to Venus easily.\footnote{Using trigonometry, try to deduce a formula on your own!}
\end{itemize}
\section{Parallex}
\begin{itemize}
\item Why it works?
\begin{itemize}
\item Very distant stars show minimal angular motion compared to nearby stars as size of angular motion is inversely proportional to distance of the star , so we can assume the pattern of
 background stars is fixed.
\end{itemize}
\item The distance at which a star has an annual parallex of 1 arcsec is defined as 1 parsec (pc)
\begin{equation}
D(pc)=\frac{1}{\theta"}
\end{equation}
For example, the nearest known star, Proxima Centauri, is at a distance of 1.30pc. The distance to Alpha Centauri is around 1.34 pc. 
\item It is very difficult to measure parallax accurately in Ground-based measurements. The Hipparcos space mission, however, enables us to measure measure parallaxs with an error or 1mas, for relatively bright stars.
\end{itemize}
\section{True luminosity}
 \begin{itemize}
 \item If we know the true luminosity\footnote{Check previous Notes!} $L$ of an object and measure its flux $F$, then we can deduce the distance. Recall that:
 \begin{align}
 F&=\frac{L}{4 \pi D^2} \\
 D&=\left(\frac{L}{4 \pi F} \right) ^{\frac{1}{2}}
 \label{Distance Formula}
 \end{align}
\item To deduce the true luminosity, we can try to locate it on the HR diagram. However, how do we find its temperature? By Wien's displacement law \footnote{ Which should be memorized at this very stage}, we can try to estimate its colour.
However, we encounter some problems
\begin{itemize}
\item Colours don't necessarily uniquely correspond to a particular temperature.
\item Colours are wrong due to dust extinction \footnote{it depends on the wavelength of light}
\end{itemize}
Besides, the distance between us and the stars is overestimated as dust reduces the flux of the stars \footnote{Is it correct to say flux of the stars?}
\item Solution: Detailed \textbf{spectroscopy}
\item Recall that Flux scales are the inverse square of distance. For example, if a thing is 2 times apart, the flux is 4 times less.
\item Check how to convert from arcsec to angles, angles to pc and pc to meters! And crack Q55.
\end{itemize}
\section{Pulsating Variables as Standard Candles}
You can read off the period of a Type 1 Cepheid and locate it to its luminosity in the log graph in figure 21. Afterwards, we can find its flux \footnote{Find out how to find the flux of a star!} and using $(\ref{Distance Formula})$, we can find the distance.
\section{Supernovae as Standard Candles}
Type I (in particular Type 1a) are “white dwarf bombs”.
this involves the conflagration of a White Dwarf exactly at its Chandrasekhar limit, the amount
of energy released is always the same, so they make excellent standard candles.
\section{Stellar Luminosity function}
\begin{itemize}
\item The stellar luminosity function $\phi(L)$ refers to the number of stars per unit
luminosity per unit volume of space
\item More luminous stars can be seen further away, so plotting a graph of number of stars/luminosity is misleading. Solutions:
\begin{itemize}
\item \textbf{Method 1}: Try to measure all stars. However, this is difficult.
\item \textbf{Method 2}: Recall that :
\begin{equation}
N(L)=\phi(L)*V_\text{max}(L)
\end{equation}
\item From $\frac{3}{2}$ law as discussed in Chapter 4, we have:
\begin{equation}
N(L) \propto  L^{\frac{3}{2}}\phi(L)
\end{equation}
\item After correcting for $V_\text{max}$ effect, We have found out that $\phi(L) \propto L^{-\alpha}$, where $\alpha=-1.35$ by observations.
\end{itemize}
\end{itemize}
\section{Stellar Mass Function}
\begin{itemize}
\item We can rearrange equations and obtain:
\begin{equation}
n(M)= \phi (L) \frac{dL}{dM}
\end{equation}
Hence, with the assumption that $L \propto M^{\beta}$
\begin{equation}
n(M) \propto L^{-\alpha} M^{\beta-1} \propto M^{-\alpha\beta+\beta-1} \propto M^{\beta(1-\alpha)-1}
\end{equation}
\item Now we plug in $\alpha=-1.35$ and $\beta=4$ and obtain $n(M) \propto M^{-2.4}$
\item However, in the coursebook, rather confusingly, we use $n(M) \propto M^{-2.35}$ and $\phi(L) \propto L^{-1.35}$
\item While \textbf{integrating}, remember that we are calculating $L \times \phi(L)dL$ and $ m \times n(m)dm$ respectively - For More \textbf{please check Exercise 59} 
\end{itemize}
\section{Lifetime of stars}
\subsection{Motivation}
\begin{itemize}
\item In Pleiades, $L^{-1.35}$ is consistent for all solar masses
\item This isn't expected for $> 5 M_\odot$
\item It shows that some original high mass stars have disappeared, why?
\end{itemize}
\subsection{Calculation}
\begin{itemize}
\item In general, burnable fraction $f$ varies along the mass range, with $f \propto M$. Considering other factors:
\begin{equation}
t_*=10.8 \left(\frac{M_*}{M_\odot} \right)^{-2} \text{Gyr}
\end{equation}
\end{itemize}
\section{Lifetime of Milky way}
\begin{itemize}
\item The fraction of all the $M_*$ stars that are still here is $\frac{t_*}{T}$ \footnote{I personally don't quite understand this statement}
\end{itemize}
\section{Hydrogen gas}
\begin{itemize}
\item In Hydrogen 1s state, there are two levels with slightly different energy levels, caused by Nuclear spin and Electron spin. The parallel state has a slightly higher energy than the anti-parallel state (think about the case of magnets)
\item The total 21cm line luminosity from a region is just proportional to the number of H atoms it contains 
\end{itemize}
\section{Molecular gas}
\begin{itemize}
\item In most of the ISM (Interstellar Medium) hydrogen is in atomic form
\item In cold dense regions the starlight can be blocked out and molecules form
\end{itemize}
\section{Dust}
\begin{itemize}
\item The energy of starlight that dust particles are absorbing
in the visible light regime has to come out somewhere else
\item Flat-slab approximation:
\begin{align}
P_\text{abs}&=A \times \frac{L}{4 \pi D^2}\\
P_\text{emit}&= A \sigma T^4
\end{align}
\end{itemize}
\section{Galaxy Types}
\begin{table}[h!]
  \begin{center}
    \begin{tabular}{c|c|c}
      \textbf{Spirals} & \textbf{Ellipticals} &\textbf{Irregulars}\\
      \hline\hline
      Flat disc with spiral arms & No disc & Disc\\
      Central bulge and spheroidal halo & Single spheroidal component & Sometimes bulge/halo\\
      Blue and red stars & Red stars only & Blue and red stars\\
      Lots of gas & No gas & Lots of gas\\   
    \end{tabular}
  \end{center}
\end{table}
\begin{itemize}
\item \textbf{Spirals} mainly form stars along the spiral arms \footnote{Trace it by the ionised gas, I forgot, please revise!}
\item \textbf{Ellipticals} stopped making stars a long time ago since the massive blue stars have all disappeared)\footnote{Since blue stars have a shorter lifespan - see chapter 7}
\item The reason is that they have no gas, which is the material needed to make new stars
\item \textbf{Irregulars} don't have gas because:
\begin{itemize}
\item Gas has not “settled down” into a disc
\item The disc has been disrupted by collision with another galaxy
\end{itemize}
\end{itemize}
\section{Spiral Discs}
\begin{itemize}
\item The brightness $I$ as a function of separation from the centre $R$ is given by:
\begin{equation}
I(R)=I_0e^{\frac{-R}{R_d}}
\end{equation}
\item Nearly all spiral discs have pretty much the same value of $I_0$, but different values
of $R_d$
THIS PART STILL INCOMPLETE!!!
\end{itemize}
\section{Ellipticals}
\begin{equation}
I(R)=I_0e^({\frac{-R}{R_e})^{\frac{1}{4}}}
\end{equation}
\section{Dark Matter}
\begin{itemize}
\item The predicted rotation curve still disagrees with the observations badly
\item The expected rotation curve is flat
\begin{itemize}
\item For increasing radius the velocity remains constant
\end{itemize}
\item How does dark matter depend on radius?
\begin{equation}
V^2=\frac{GM}{R^2}
\end{equation}
\item Therefore $M_\text{dark}(<R) \propto R$
\begin{itemize}
\item The $(<R)$ part is an obvious consequence of Birkhoff's theorem from General Relativity\footnote{See Chapter 1}
\end{itemize}
\item Considering the density $(\rho_\text{dark})$:
\begin{align}
\rho_\text{dark}&=\frac{M_\text{dark}}{R^3}\\
& \propto \frac{1}{R^2}
\end{align}
\item Therefore dark Matter is concentrated near the center
\item A large galaxy like the Milky Way has about $10^{11}M_\odot$ in stars, and $10^{12}M_\odot$ in dark matter
\end{itemize}
\section{Random star motions and interactions in spheroids}
\begin{itemize}
\item Elliptical galaxies do not rotate.
\item What is the chance of star collisions?
\end{itemize}
\subsection{Probability}
\begin{itemize}
\item Given by the ratio of $\pi{R_\text{sep}}^2$ and $\pi R^2$, \textbf{where $R=2R_*$ and $R_*$ is the radius of the stars} \footnote{ I think $R=2R_*$ because we are considering the diameter- I'm not completely sure!}
\begin{align}
p={\left(\frac{R}{R_\text{sep}}\right)}^2\\
={\left(\frac{2R_*}{R_\text{sep}}\right)}^2 
\end{align}
\textbf{The second line only applies if they have the same size}
\end{itemize}
\subsection{Time between collisions}
\begin{itemize}
\item Consider a cylinder covering the path of the stars
\item As a convention, we let $\sigma$ as the cross-sectional area of the cylinder
\item Consider the number density of the stars is $n$, then the number of stars in the cylinder:
\begin{equation}
N=n \sigma vt_\text{collision}
\end{equation}
We set $N=1$ and hence:
\begin{equation}
t_\text{coll}=\frac{1}{n \sigma v}
\end{equation}
\item After substitution, we find out $t_\text{coll} > \text{age of universe}$
\section{Gravitational Radius $R_\text{grav}$}
The radius in which the kinetic and potential energy of the passing star are equal in magnitude is defined as the $R_\text{grav}$. For $R_\text{star}<R_\text{grav} $, its path is deflected by the other star. Assuming both stars have the same mass $m$:
For potential energy:
\begin{equation}
U=\frac{-Gm^2}{R}
\end{equation}
For kinetic energy:
\begin{equation}
K=\frac{mv^2}{2}
\end{equation}
Taking their magnitudes:
\begin{align}
\frac{Gm^2}{R}&=\frac{mv^2}{2}\\
2Gm&=Rv^2\\
R_\text{grav}&=\frac{2Gm}{v^2}
\end{align}
\end{itemize}
\section{Galaxy collisions}
\begin{itemize}
\item Stars almost never collide, but galaxies do
\item Galaxy interactions make strange non-symmetric disturbed structures as they pass by each other
\end{itemize}
\section{Active Galactic Nuclei}
\subsection{Luminosity}
\begin{itemize}
\item All galaxies except irregulars are brightest at the centre. However a minority have an especially bright central spot
\item Example: Quasars
\item Some galaxies can have nuclear luminosities which are around $10\%$ of the luminosity of an entire normal galaxy of stars like the Milky Way
\item Do exercise \text{73}
\end{itemize}
\subsection{Board emission lines}
\begin{itemize}
\item AGN emission lines are very broad. The $H\alpha$ line at 656.3nm is about 20nm across.
\item The broadening of the emission lines must be due to internal gas motions
in the nucleus
\item See the notes for Doppler Shift!
\end{itemize}
\subsection{Multi-wavelength spectral energy distribution}
\begin{itemize}
\item AGNs radiate strongly at all sorts of wavelengths
\item The feature which dominates the energy output for most
AGN output however is the “Big Blue Bump”, which peaks in the UV 
\item There are secondary peaks of emission in the mid-infrared (at a wavelength of around 10$\mu $m), and in the X-rays.
\item \textbf{UV}
\begin{itemize}
\item  It peaks around $\lambda = 0.03 \mu m$ suggesting $T \approx 10^5K$
\item  However, the UV bump looks broader than a single blackbody
\item  It is variable 
\end{itemize}
\item \textbf{IR}
\begin{itemize}
\item   It peaks around $\lambda = 10 \mu m$ suggesting $T \approx 10^5K$
\item The IR bump is almost certainly due to dust re-radiation, but the dust is hotter than the dust we normally see in the interstellar medium in normal galaxies, which is it at $T \approx 50K$
\end{itemize}
\item \textbf{X-rays}
\begin{itemize}
\item  It is almost certainly not due to blackbody radiation, but if the emission is due to hot gas in some way, it must
have a wide range of temperatures, $10^{7-9}K$.
\end{itemize}
\end{itemize}
\section{Variability of AGNs}
\begin{itemize}
\item Vary erratically, cannot be used like Cepheids
\item There is some disturbance transmission mechanism (some sort of waves). The transmittion time:
\begin{equation}
\Delta t =\frac{R}{v_d}
\end{equation}
\item Waves are transmitted by particle collisions
\item If we equate the Thermal Energy:
\begin{equation}
E=\frac{3kT}{2}
\end{equation}
with the Kinetic Energy:
\begin{equation}
K=\frac{mv^2}{2}
\end{equation}
\item We obtain:
\begin{align}
\frac{3kT}{2}&=\frac{mv^2}{2}\\
3kT&=mv^2\\
v&=\sqrt{\frac{3kT}{m}}
\end{align}
\item Assume $T=10^5k$ and we obtain 49.8 $km s^{-1}$\textbf{, and according to Astrophysicists\footnote{The assumptions in all the calculations are too approximate for my liking!}, 31.7 $\approx$ 49.8, so we can see it probably comes from sound waves	}\footnote{31.7 $kms^{-1}$ is obtained in \textbf{Exercise 77}}
\end{itemize}
\section{Motion around AGN}
\begin{itemize}
\item Recall our equation:
\begin{equation}
v^2=\frac{GM}{R}
\end{equation}
\item Most of the time we only see a single unresolved spot 
\item However, for some of the closest AGN, we can see glowing gas which is ionised by the AGN at large enough distances to visually separate from the central spot
\item Need to correct the angle since movements along the plane of sky (as opposed to plane of sight) don't induce Doppler's shift
\begin{itemize}
\item Solution: Observe the image on the sky, if it looks like a ellipse
\end{itemize}
\end{itemize}
\section{Emission line widths and time lag}
\begin{itemize}
\item Some gases emit lines, they are ionised by the time-varying UV light
\item Time lag $t_\text{lag}$ exists since light need time to travel from centre of AGN to the gas
\begin{equation}
R_\text{Broad line region}=ct_\text{lag}
\end{equation}
Afterwards, we can obtain $V_\text{BLR}$ by estimating the line width. The mass $M$ can be estimated by:
\begin{equation}
M=\frac{R_\text{BLR}V_\text{BLR}^2}{G}
\end{equation}
\end{itemize}
\section{AGN masses and luminosity}
\begin{itemize}
\item Stars range over about a factor of $10^9$ (1 billion) in luminosity, but only about a factor of $500$ in mass
\item Similarly, AGNs range over about a factor of $10^6$ (1 million) in luminosity, but only about a factor of $1000$ in mass
\item In AGNs, there is little correlation between mass and luminosity, as opposed to stars
\end{itemize}
\section{Dark masses}
\begin{itemize}
\item By measuring the width of absorption lines in the light from a galaxy, we can estimate the speed of random motions of the stars in that galaxy \footnote{How is it linked to "broad emission lines" in the section above?}
\item If we make such a measurement at a series of different locations within the galaxy, we can see that stellar velocities start to increase rapidly towards the centre of most galaxies
\item From the methods introduced above, we can deduce the mass in the centre of AGN, which is far less than the stellar mass estimated from starlight 
\item We deduce the "missing mass" is dark mass
\end{itemize}
\section{Power from black hole accretion}
\begin{itemize}
\item If a small mass $\Delta m$ starts from a distance and falls onto a large mass $M$, it gains kinetic energy:
\begin{equation}
E=\frac{GM \Delta m}{R}
\end{equation}
\item For gases however, friction and collisions randomise the energy and turn it into heat
\item For a black hole:
\begin{equation}
R_\text{Event Horizon}=\frac{2GM}{c^2}
\end{equation}
\item By substitution,
\begin{align}
E&=\frac{GM \Delta m}{\frac{2GM}{c^2}}\\
&=\frac{1}{2}\Delta m c^2
\end{align}
\item The efficiency, $\mu$, is 0.5, which is much higher than $\mu_\text{nuc}=0.007$
\item This is obviously, a fallacy
\item \textbf{Incorrect Assumptions}
\begin{itemize}
\item Radial free fall is unlikely
\item At larger distances, the
accreting material will always have some angular momentum, and end up forming a rotating disc around the black hole
\item Friction between neighbouring radial annuli then allows
the material to slowly spiral inwards, forming a gradually heated accretion disc
\item Efficiency $\mu$ is reduced by the following 2 effects:
\begin{itemize}
\item Effective radius is \textbf{not} the Event Horizon. For non-rotating black-holes, the ISCO is located at:
\begin{align}
r_\text{ms}&=6\frac{GM}{c^2}\\
&=3R_\text{EH}
\end{align}
\item I.e. There is no stable orbit $<3R_\text{EH}$
\item And therefore the thermal energy gained by the gas decreases
\item Some energy (half) is converted to rotational energy, as:
\begin{align}
\Delta K& =\dfrac{dK}{dR} \Delta R\\
&=-\frac{GM\Delta m}{2R^2} \Delta R
\end{align}
\begin{align}
\Delta U& =\dfrac{dU}{dR} \Delta R\\
&=-\frac{GM\Delta m}{R^2} \Delta R
\end{align}
\begin{align}
\frac{\Delta K}{\Delta U}&=\frac{-\frac{GM\Delta m}{2R^2} \Delta R}{-\frac{GM\Delta m}{R^2} \Delta R}\\
&=\frac{1}{2}
\end{align}
\end{itemize}
\end{itemize}
Combining with other effects, we have $\mu \approx 0.1$
\end{itemize}
\section{Big Bang}
\begin{itemize}
\item Consider a galaxy on the edge of a sphere in the universe centered on us
\item By Birkhoff's theorem
\begin{itemize}
    \item This galaxy is effected gravitationally only by the mass within the sphere
    \item The mass within a sphere acts like a point mass at the centre
\end{itemize}
\item Hence the energy of the galaxy:
\begin{equation}
    E=\frac{mV^2}{2}-\frac{GMm}{r}
\end{equation}
\item \textbf{If $E<0$, the universe re-collaspes in a big crunch}
\item \textbf{If $E>0$, the universe continues to expand}
\end{itemize}
\section{Critical Density of the Universe}
\begin{itemize}
    \item Consider the equations below:
    \begin{align}
        \frac{mV^2}{2}&=\frac{GMm}{r}\\
    M&=\frac{4 \pi r^3}{3} \rho_\text{crit}
    \end{align}
    We can find $\rho_\text{crit}$ by combining the equations and using $V=H_0r$
    \begin{align}
        \frac{mV^2}{2}&=\frac{4Gm\pi r^3 \rho_\text{crit}}{3r}\\
         3V^2&=8G\pi r^2 \rho_\text{crit}\\
         3{H_0}^2{r^2}&=8G\pi r^2 \rho_\text{crit}\\
          3{H_0}^2&=8G\pi\rho_\text{crit}\\
         \rho_\text{crit}&=\frac {3{H_0}^2}{8G\pi}
    \end{align}
    \item As $\rho_\text{crit}$ decrease with increasing $r$, by referring to the equation above, we can proof that $H_0$ changes with time
\end{itemize}
\section{Cosmic Scale Factor}
\subsection{Definition}
\begin{equation}
    R(t)=\frac{r(t)}{r(t=\text{today})}
    \end{equation}
    Note that it can be rewritten as:
\begin{equation}
    R(t_\text{emit})=\frac{\lambda_\text{emit}}{\lambda_\text{obs}}
\end{equation}
\subsection{Redshift}
\begin{equation}
z=\frac{\lambda_{\circ}-\lambda_e}{\lambda_e}
\end{equation}
And hence:
\begin{equation}
    R(t)=\frac{1}{1+z}
\end{equation}
\section{Geometry of the universe}
We define the density parameter as:
\begin{equation}
    \Omega_0=\frac{\rho}{\rho_\text{crit}}
\end{equation}
\begin{itemize}
    \item Closed (the big crunch): $\Omega>1$, $E<0$
    \item Open(Expanding):$\Omega<1$,$E>0$
    \item Flat: $\Omega=1$,$E=0$
\end{itemize}
\section{Problems of the Big Bang Model}
\begin{itemize}
    \item Flatness Problem
    \begin{itemize}
        \item The fine-tuning of density such that it is so close to the critical density
    \end{itemize}
    \item Horizon Problem
    \begin{itemize}
    \item Two ends of the horizon looks the same and has the same temperature, which implies they must have been connected. By def, as our horizon is the furthest light can travel since the big bang, these two regions (in this case separated by 2 horizon scales) can not have been connected at the big bang
\end{itemize}
\end{itemize}
\section{Guth Inflation Solution}
\begin{itemize}
    \item Alan Guth developed a theory called \textbf{Inflation} where a new form of energy from a field created roughly $10^{-30}$ seconds after the big bang accelerated the
expansion of the Universe at speeds faster than light 
    \item How it solved the aforementioned problems?
    \begin{itemize}
        \item Flatness Problem:
        \begin{itemize}
            \item  The rapid acceleration would flatten out any irregularity in the geometry of the early post-big-bang Universe
        \end{itemize}
        \item Horizon Problem:
        \begin{itemize}
            \item The Universe was all connected before inflation
        \end{itemize}
    \end{itemize}
\end{itemize}
\section{Energy density of matter and radiation}
\begin{align}
\rho_\text{m} &\propto {R(t)}^{-3}\\
\rho_\text{r} &\propto {R(t)}^{-4}
\end{align}
\begin{itemize}
\item At the start of the universe, $R(t)$ was small, and the universe was dominated by radiation
\end{itemize}
\section{Beginning of the universe}
\begin{itemize}
\item This era was described as "opaque" as the light from this era cannot travel out into the Universe ,\textbf{because the photons scatter off electrons}
\item The particles in the early Universe are not distributed randomly and
there are fluctuations due to tiny random \textbf{quantum fluctuations}
in the pea-sized Universe that existed after the big bang and before $10^{-30}s$
\end{itemize}
\section{Recombination}
\begin{itemize}
\item In the radiation dominated Universe, the photons are so dense that the
photon pressure can push the clumps apart. An ongoing battle between matter trying to clump
and photon pressure pushing these apart can be observed in the first view of the Universe at
\textbf{Recombination}.
\item Recombination is the era where the energy density of matter now dominates and the charged
particles have combined to form neutral atoms so the photons freely travel out into the expanding
Universe. We detect the first radiation from this era in the form of the cosmic microwave
background.
\end{itemize}
\section{Chemical building blocks}
\begin{itemize}
\item In the very early radiation dominated Universe, protons and neutrons move fast, collide often and fuse to form nuclei. The high density of high energy gamma ray photon however manage to
blast any nuclei that form apart $\implies$ None are formed
\item In a later stage, the expansion has reduced the photon energy to such an extent that they are unable to break the more stable nuclei up, while the Universe is dense and hot
enough for particle collisions and nuclear fusion to occur $\implies$ Starts to form
\item At the end of the nucleosynthesis period the Universe has cooled to such an extent that no
heavier elements can be made as the density decreases (with the expansion) the collisions
become rare and unimportant. The rest of the chemicals on our periodic table are created much
later on in the history of the Universe in the cores of stars \footnote{You have learnt them before!}
\end{itemize}
\section{Deuterium}
\begin{itemize}
\item Heavy Hydrogen 
\item Why important?
\begin{itemize}
\item There is no way to produce any quantity of this element at
any other point in the history of the Universe except during nucleosynthesis.
\item  The cores of stars have the required temperatures for Deuterium to be created, but the stars are so dense, the chain of reactions don’t stop and they use up all of their Deuterium in other reactions
\end{itemize}
\end{itemize}
\section{Nucleosynthesis and the baryon density}
\begin{itemize}
\item Protons, neutrons, everything in the periodic table is baryons
\item We define the baryon density parameter as:
\begin{equation}
\Omega_b=\frac{\rho_b}{\rho_\text{crit}}
\end{equation}
\item From nucleosynthesis observation we measure $\Omega_b$ = 0.045
\end{itemize}
\section{Einstein Ring}
\section{Cosmic Microwave Background}
\begin{itemize}
\item If the Universe started in a hot big bang, then the atoms would have interacted so strongly that all detailed features in their energy distribution are washed out and we would expect to see a thermal continuous blackbody spectrum from the Big Bang
\item Experimental data fits extremely well with blackbody curve for $T=2.74k$, which confirms our theory on the competition between matter and radiation in the early Universe
\end{itemize}
\section{What is Dark Matter}
\begin{itemize}
\item An early hypothesis was that this dark matter was cold baryonic matter bound up in dark objects such as failed faint Brown Dwarf stars
\item We know from nucleosynthesis that only $\approx$ 5\% of the Universe is made up of baryons, so whatever this extra mass is it must be
non-baryonic
\item Our best guess is that it's \textbf{cold non-baryonic massive particles}
\item Modified Newtonian Dynamics uses $F=ma^2$
\begin{itemize}
\item Need a tuning parameter which changes for different galaxies
\item Goes against the Cosmological Principle that everything (including the law
of physics) is homogenous and isotropic
\end{itemize}
\end{itemize}
\section{Distance-redshift relation}
\begin{itemize}
\item This section resulted in a Nobel Prize in 2011. They used distant supernova as standard candles to probe the Universe
\item We have the formula $D=\frac{cz}{H_0}$. There is an extension that works for more distant galaxies:
\begin{equation}
D_L \approx \frac{c(1+z)(z-\frac{1+q}{2}z^2)}{H_0}
\end{equation}
\item Where $q=-\left(1+\frac{\dot{H}}{H^2}\right)$ and $\dot{H}$ is the rate of change of the Hubble parameter
\end{itemize}
\section{Concordant cosmology}
\begin{itemize}
\item Baryonic content  $\Omega_b \approx 0.05$
\item The total dark and baryonic matter content $\Omega_m \approx 0.3$
\item Dark energy content $\Omega_\Lambda \approx 0.7$
\item The university is flat and has critical density:  $\Omega_b+\Omega_m+\Omega_\Lambda=1$
\end{itemize}
\end{document}

